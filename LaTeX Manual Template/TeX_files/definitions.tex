\newcommand{\andspace}[1]{\hspace{#1}\textrm{and}\hspace{#1}}

\numberwithin{equation}{section}
\setlength{\columnsep}{.5cm}
\setlength{\columnseprule}{1pt}
\def\columnseprulecolor{\color{black}}

\newcommand{\abs}[1]{\left| #1 \right|}
\newcommand{\inner}[1]{\langle #1 \rangle}
\newcommand{\norm}[1]{\left\lVert#1\right\rVert}
\newcommand{\spanvect}{\textnormal{span}}
\newcommand{\union}{\cup}
\newcommand{\Union}{\bigcup}

%create a section without making the section title.
\newcommand\invisiblesection[1]{%
	\refstepcounter{section}%
	\addcontentsline{toc}{section}{\protect\numberline{\thesection}#1}%
	\sectionmark{#1}}

%Makes a chapter with no title
\makeatletter
\newcommand{\unchapter}[1]{%
	\begingroup
	\let\@makechapterhead\@gobble % make \@makechapterhead do nothing
	\chapter{#1}
	\endgroup
}
\makeatother

% Define the questions environment
\newenvironment{questions}{%
	\begin{tcolorbox}[
		colback=orange!10,
		colframe=orange!50,
		arc=0pt,
		boxrule=1pt,
		title={\faQuestion\ Questions},
		fonttitle=\bfseries,
		coltitle=blue!50!black,
		colbacktitle=yellow!20,
		attach title to upper={\par},
		boxsep=0pt,
		]
		\begin{itemize}  % Start the itemize environment
			\BODY  % This will contain the list of questions
		\end{itemize}  % End the itemize environment
	\end{tcolorbox}
}

% define interesting note
\newenvironment{interestnote}{
	\begin{tcolorbox}[
		colback=green!10,  % Very light green background color of the box
		colframe=green!50,  % Border color of the box
		arc=0pt,  % Adjust the corner radius of the box
		boxrule=1pt,  % Border thickness
		title={\faLightbulb\ Interesting Note},  % The lightbulb icon and title
		fonttitle=\bfseries,  % Font style for the title
		coltitle=blue!50!black,  % Color for the title text
		colbacktitle=yellow!20,  % Background color for the title
		attach title to upper={\par},
		boxsep=0pt,  % Adjust the space between the content and the box
		]
	}{
	\end{tcolorbox}
}

\definecolor{lightpurple}{RGB}{220,180,240}

\newenvironment{quotationbox}{
	\begin{tcolorbox}[
		colback=lightpurple!10,  % Very light purple background color of the box
		colframe=lightpurple!50,  % Border color of the box
		arc=0pt,  % Adjust the corner radius of the box
		boxrule=1pt,  % Border thickness
		title={\faQuoteLeft\ Quotation},  % The quotation icon and title
		fonttitle=\bfseries,  % Font style for the title
		coltitle=blue!50!black,  % Color for the title text
		colbacktitle=yellow!20,  % Background color for the title
		attach title to upper={\par},
		boxsep=0pt,  % Adjust the space between the content and the box
		]
	}{
	\end{tcolorbox}
}

% define a URL
\newenvironment{urlbox}{
	\begin{tcolorbox}[
		colback=blue!10,  % Very light green background color of the box
		colframe=blue!50,  % Border color of the box
		arc=0pt,  % Adjust the corner radius of the box
		boxrule=1pt,  % Border thickness
		title={URL},  % The lightbulb icon and title
		fonttitle=\bfseries,  % Font style for the title
		coltitle=blue!50!black,  % Color for the title text
		colbacktitle=yellow!20,  % Background color for the title
		attach title to upper={\par},
		boxsep=0pt,  % Adjust the space between the content and the box
		]
	}{
	\end{tcolorbox}
}

\newcommand{\unfinished}{%
	\par\noindent%
	\setlength{\fboxsep}{10pt} % Adjust the padding
	\fcolorbox{red}{red!20}{%
		\begin{minipage}{\dimexpr\linewidth-2\fboxsep}%
			\vspace{5pt} % Adjust the vertical spacing
			\centering
			\textcolor{red}{\textbf{\faExclamation\ Section in Progress\ \faExclamation}}
			\vspace{5pt} % Adjust the vertical spacing
		\end{minipage}%
	}%
	\par
}

% Define Python style
\lstdefinestyle{pythonstyle}{
	language=Python,
	basicstyle=\ttfamily\small,
	numbers=left,
	numberstyle=\tiny\color{gray},
	frame=single,
	rulecolor=\color{black},
	keywordstyle=\color{blue},
	commentstyle=\color{green!40!black},
	stringstyle=\color{red},
	breaklines=true,
	tabsize=4,
	captionpos=b,
	extendedchars=true,
	inputencoding=utf8,
	showspaces=false,
	showstringspaces=false,
	xleftmargin=\parindent,
	xrightmargin=\parindent
}

\lstdefinestyle{shellstyle}{
	language=sh,                   % Set the language to Shell
	basicstyle=\ttfamily,           % Use a monospaced font
	backgroundcolor=\color{gray!10},% Background color
	keywordstyle=\color{blue},      % Keywords are blue
	commentstyle=\color{green!40!black}, % Comments are green
	stringstyle=\color{red},        % Strings are red
	breakatwhitespace=false,        % Don't break lines at whitespace
	breaklines=true,                % Automatically break long lines
	postbreak=\mbox{\textcolor{red}{$\hookrightarrow$}\space}, % Line continuation symbol
	showstringspaces=false,         % Don't show spaces in strings
	captionpos=b,                   % Caption position (bottom)
	frame=single,                   % Add a frame around listings
	numbers=left,                   % Line numbers on the left
	numberstyle=\tiny\color{gray},  % Line number style
	stepnumber=1,                   % Line number increments
	tabsize=4,                      % Tab size
	xleftmargin=15pt,               % Left margin
	xrightmargin=0pt,               % Right margin
}

% Define the custom terminal style
\lstdefinestyle{terminalstyle}{
	language=bash,                   % Set the language to Bash
	basicstyle=\ttfamily\color{white}, % Use a monospaced font and white text
	backgroundcolor=\color{black},   % Background color
	frame=tb,                        % Top and bottom frame lines
	framerule=0.5pt,                 % Frame rule width
	xleftmargin=10pt,                % Left margin
	xrightmargin=10pt,               % Right margin
	rulecolor=\color{blue!70},       % Frame color
	showstringspaces=false,          % Don't show spaces in strings
	upquote=true,                    % Use straight quotes
	commentstyle=\color{white},      % Comments are white
	keywordstyle=\color{white},      % Keywords are white
	numbers=left,                    % Line numbers on the left
	numberstyle=\tiny\color{gray},   % Line number style
	captionpos=b,                    % Caption position (bottom)
	breaklines=true,                 % Automatically break long lines
}
